\documentclass[preprint]{iucr}
 \journalcode{J}
 \papertype{CP}

\usepackage{graphicx}

\usepackage[T1]{fontenc}
\usepackage[utf8]{inputenc}

\begin{document}

\title{The Fast Azimuthal Integration Python library}
\shorttitle{PyFAI}

    \author[a]{Aurore}{Deschildre}
    \author[a]{Giannis}{Ashiotis}
    \author[b]{Zubair}{Nawaz}
    \author[a]{Jonathan P.}{Wright}
    \author[c]{Fr\'ed\'eric-Emmanuel}{Picca}
    \cauthor[a]{J\'er\^ome}{Kieffer}{jerome.kieffer@esrf,fr}{}
    \aff[a]{European Synchrotron Radiation Facility, \city{Grenoble}, \country{France}}
    \shortauthor{Kieffer et al.}

\maketitle

\begin{synopsis}

\end{synopsis}

\begin{abstract}
\end{abstract}

\section{Introduction}

\subsection{Experiment description}
In pyFAI, a basic experiment is defined by an area-detector which position in
space is defined versus the intersection of the incoming pencil beam and the
sample plan. 

\subsection{Detector}
Like most other image processing packages, pyFAI allows the definition of 2D
detectors by the pixel size (in meter) but shows its limits with modern pixel
detectors.

\subsubsection{Detectors classes}
Large area pixel detector are often composed from the assembly of smaller
modules. Often such detector exhibit gaps beteween modules and pixels of various
sizes within a module, hench requires a specific mask.
To take the specificites of each detector into account, pyFAI contains about
40 detectors class definition which contains the mask (invalid pixels, gaps,
\ldots) and a method to calculate the pixel positon in cartesian coordinates. 

\subsubsection{Nexus Detectors}
Optically coupled CCD detector and Pilatus detectors should also be corrected
for small spacial displacement of the pixel, often called geometric distortion. 
Any detector object in PyFAI, can be saved to a HDF5 file
following the NeXus conversion (slightly modified).
Detector objects can then be restored from the disk, making detector definition
less error-prone.
Pixels of a 2D detector are saved as a 4D HDF5-dataset: 2D array of vertices
pointing to every corner of each pixel, which makes an array of shape: (ny, nx, 4, 3)
This can define detectors no longer planar, like tiled detectors from 
ImXpad or the Pilatus12M from Dectris, even accounting for the normal of each
pixel.

\subsection{Geometry}

\subsubsection{Binning}
One of the strength of this geometry is capability to perform binning of the
detector without having to re-calibrate or re-calculate the position in space.
All pyFAI detector classes have a binning option which will increase the pixel
size accordingly and divide detector shape accordingly.
\section{Calibration}
\subsection{Calibrant}
\subsection{Peak-picking}
\subsubsection{Massif}
\subsubsection{blob detection}
\subsection{User interface}
\section{Integration}
\subsection{Pixel splitting schemes}
\subsubsection{No splitting}
\subsubsection{Bounding box splitting}
\subsubsection{Tight pixel splitting}
\subsubsection{Pixel correlation}
\subsection{User interface}
\section{Other features}
\subsection{Forward projection}
\subsection{Image offset}
\subsection{amorphous background removal}

\section{Conclusion}




\bibliographystyle{iucr}
\bibliography{biblio}
\appendix
\section{Project structure}
PyFAI is an open source project licensed under the GPL mainly written in Python (v2.6 or 2.7) 
and heavily relying on the python scientific ecosystem: NumPy, SciPy and Matplotlib. 
It provides high performances image treatment thanks to Cython and PyOpenCL.

The project is hosted on GitHub (https://github.com/kif/pyFAI) which provides
the issue tracker in addition to code hosting. 
To ease the distribution, the
software is available as official Debian package and included in some of the
most famous linux distribution like Ubuntu and Debian. 
The program runs also on other operating systems like Windows and MacOSX. 

Anybody can fork the project and adapt it to his own needs: CEA-saclay,
Synchrotrons Soleil and APS did it. If developments are useful to be shared,
they can be merged into the main branch via pull-requests.

\section{Parallel implementations using OpenCL}
Azimuthal integration and many other computing parts in pyFAI were written in
OpenCL kernels and interfaced using PyOpenCL \cite{pyopencl}. PyOpenCL offers a
shared execution model effective both on processors (CPU), graphics cars (GPU)
and accelerators like FPGA.

\subsection{Azimuthal Integration}
The direct azimuthal integration is basically a scatter operation which 
requires large locked section. 
To overcome this limitation, pixels have been
associated to the output bin of the histogram and stored in a look-up
table (LUT), making the integration look like a simple (if large and sparse)
matrix vector product.
The sparse matrix ``Compressed Sparse Row'' (CSR) format is now used in pyFAI,
saving about half of the space of the LUT previously used.
Secondly, all threads within a workgroup are collaborating to calculate the
matrix-vector product via a so-called ``parallel reduction'', offering
additional speed-up (especially on GPUs).
The compensated algebra (Kahan summation) is kept to maintain the precision 
of the calculation while using single precision (32 bits) floating points
arithmetics.

\subsection{Look-up table creation}


\ack{Acknowledgements}

The author would like to thank all ESRF beamline teams for \ldots 

In the instrumentation division (ISDD) we would like to thank Claudio Ferrero,
head of data analysis unit, and Andy G\"otz, head of software group, for
supporting the parallelization work on the presented algorithm.
V. Armando Solé, the author of PyMca, is also acknowledged for providing a

LinkSCEEM \ldots



\end{document}
